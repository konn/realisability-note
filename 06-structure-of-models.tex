%#!luajitlatex -src-specials 06-structure-of-models.tex

\documentclass[realisability.tex]{subfiles}

\begin{document}
\section{$V^{\mathcal{R}}$の構造}
本節ではKrivine~\cite{Krivine:2014yu}の内容を再整理しつつ,実現可能性モデル$V^{\mathcal{R}}$の構造について細かく分析していく.

\subsection{$V$を初等拡大する$\hat{2}$-値構造}
$\hat{2}$は関数$\truth{ \varphi[-] }$によって$V$の情報をコードしている.
実際,$V^{(\mathcal{R})}$の場合は$V^{(\mathcal{R})}$上に$\gimel 2$-値構造が入り,これがあたかも$V$の初等拡大のように振る舞う.
この状況は$V^{(\mathcal{R})}$という同じドメイン上に異なる構造が載っており,直観が掴みづらい.
我々の$V^{\mathcal{R}}$では,上で定義した$V^{\mathcal{R}}$の部分クラス$\hat{V} \subsetneq V^{\mathcal{R}}$が$V$の初等拡大のように振る舞うので,随分と見通しがよくなる.

まず,$\truth{ - }$の値が自然な形で計算できる事を見る:
\begin{lemma}\label{lem:char-homo}
 $\truth{ - }: \hat{V}^n \to \hat{2}$は論理式の関数として見た時,Boole代数の準同型のように振る舞う.即ち:
 \begin{gather*}
  V^{\mathcal{R}} \models \truth{\bot} = 0, \quad \forall \vec{x} \strin \hat{V}\: \truth{\varphi[\vec{x}] \to \psi[\vec{x}]} = - \truth{\varphi[\vec{x}]} + \truth{\psi[\vec{x}]},\\
  V^{\mathcal{R}} \models \forall \vec{y} \strin \hat{V} \: \truth{\forall x \: \varphi[x, \vec{y}]} = \prod_{x \in \hat{V}} \truth{\varphi[x, \vec{y}]}.
 \end{gather*}
\end{lemma}
\begin{proof}
 論理式の構成に関する帰納法で示す.
 $\truth{\bot} = 0$は明らか.

 また,基礎モデルにおいては$\truth{ \varphi[x] \to \psi[x] } = -\truth{\varphi[x]} + \truth{\psi[x]}$はどんな$x$についても成り立つから,等式に関する絶対性より
 \[
  V^{(\mathcal{R})} \models \forall x \strin \hat{V} \: \truth{ \varphi[x] \to \psi[x] } = -\truth{\varphi[x]} + \truth{\psi[x]}
 \]
 は常に成立する.

 最後に全称量化の場合を考える.
 次の二つがそれぞれ実現されていることを見ればよい:
 \begin{enumerate}
  \item $\forall x, z \strin \hat{V} \: \truth{\forall y \: \varphi(x, y) } \leq \truth{\varphi(x, z)}$,
  \item $\forall x \strin \hat{V} \: \forall \alpha \strin \gimel 2\:
          \left[\forall z \strin \hat{V} \: \alpha \leq \truth{\varphi(x, z)}
            \implies \alpha \leq \truth{\forall y \: \varphi(x,y)}
          \right]$.
 \end{enumerate}
 一つ目の条件,下界である事については$\comb{I}$が実現することがすぐにわかる.
 二つ目の条件も,実は$\comb{I}$が実現する.
 これを見るため,$x \in V$と$\alpha < 2$を任意に固定し,代入結果がすべて$\comb{I}$で実現されている事を見よう.

 $\alpha = 0$の時は,
 \[
  \|\forall z \strin \hat{V} \: 0 \leq \truth{\varphi(\hat{x}, z)}
 \to 0 \leq \truth{\forall y \: \varphi(\hat{x},y)}\| = \|(\bot \to \bot) \to (\bot \to \bot)\|
 \]
 なので,$\comb{I} \Vdash (\bot \to \bot) \to (\bot \to \bot)$よりOK.

 最後に$\alpha = 1$の時を考える.
 \[
  \|\forall z \strin \hat{V} \: 1 \leq \truth{\varphi(\hat{x}, z)}
 \to 1 \leq \truth{\forall y \: \varphi(x,y)}\| =
 \|\forall z \strin \hat{V} \:[\truth{\varphi(\hat{x}, z)} = 1]
 \to \truth{\forall y \: \varphi(\hat{x},y)} = 1\|
 \]
 よって,$\xi \Vdash \forall z \strin \hat{V} \: \truth{\varphi(\hat{x},z)} = 1$と$\pi \in \|\truth{\forall y \: \varphi(\hat{x},y)} = 1\|$を任意に取って$\comb{I} \cons \xi \push \pi \in \pole$を示す.
 ここで$V \models \forall y \: \varphi(x,y)$か否かで場合分けする.

 $V \models \forall x \: \varphi(x, y)$の時は,$\truth{x, z} = 1$が任意の$z \in V$について成立しているから,
 \[
 \left|\forall z \strin \hat{V} \: \truth{\varphi(x,z)} = 1\right| = \bigcap_{z \in V} \left|\truth{\varphi(\hat{x}, \hat{z})} = 1\right| = | \bot \to \bot |.
 \]
 一方で仮定より$\pi \in\|\truth{\forall y \: \varphi(x, y)} = 1\| = \| \bot \to \bot \|$なので$\comb{I} \cons \xi \push \pi \reds \xi \cons \pi \in \pole$となる.

 $V \not\models \forall x \: \varphi(x)$の時を考える.
 この時$V \models \neg \varphi(y_0)$なる$y_0$を取っておけば,$\truth{\varphi(x, y_0)} = 0$であり,
 \[
 \left|\forall z \strin \hat{V} \: \truth{\varphi(\hat{x},\hat{z})} = 1\right| = \bigcap_{z \in V} \left|\truth{\varphi(\hat{x}, z)} = 1\right| \subseteq |\truth{\varphi(\hat{x}, \hat{y}_0)} = 1| = | \top \to \bot |.
 \]
 また仮定より$\|\truth{\forall y\: \varphi(x, y)} = 1\| = \|\top \to \bot\|$.
 よってこの場合も$\comb{I}$が実現してくれる事がわかる. \qed
\end{proof}
また,定義に戻って展開すれば直ちにつぎがわかる:
\begin{lemma}
 $V^{\mathcal{R}} \models \forall x, y \strin \hat{V}\: [ \braket{x \in y} = 1 \iff x \strin y],\quad \forall x, y \strin \hat{V}\:[\braket{x = y} = 1 \iff x = y]$.
\end{lemma}
個別の$\hat{\;}$-名称については,強い関係と外延的関係は一致する:
\begin{lemma}
 $\mathcal{R}$が斉一的で$x, y \in V$とする.
 \[
  V^{\mathcal{R}} \models \quoted{\hat{x} \in \hat{y}} \iff V \models x \in y, \qquad
  V^{\mathcal{R}} \models \quoted{\hat{x} \subseteq \hat{y}} \iff V \models x \subseteq y.
 \]
 よって特に$V^{\mathcal{R}} \models \hat{x} \simeq \hat{y} \iff V^{\mathcal{R}} \models \hat{x} = \hat{y}$.
\end{lemma}
\begin{remark}
 上の結果は$V^{\mathcal{R}} \models \quoted{\hat{V}: \text{外延的クラス}}$を意味しない!
 実際,$\hat{2}$が非自明な場合$\hat{V}$には$\check{x}$の形で書けないアトムが複数存在する.
\end{remark}
以上を踏まえると,$V$から$\hat{V}$への「初等埋め込み」は次の形で定式化出来る:
\begin{corollary}
 $V^{\mathcal{R}} \models \quoted{(\hat{V}, \truth{{\cdot} \in {\cdot}}, \truth{{\cdot} = {\cdot}}): \ZF\text{ の }\hat{2}\text{-値モデル}}$.

 また$\varphi[x_1, \dots x_n]$を$\ZF$-論理式,$a_i \in V$とすると
 \[
  V \models \varphi[a_1, \dots, a_n] \iff V^{\mathcal{R}} \models \quoted{(\hat{V}, \truth{{\cdot} \in {\cdot}}, \truth{{\cdot} = {\cdot}}) \models \varphi[\hat{a}_1, \dots \hat{a}_n]}.
 \]
\end{corollary}
\begin{proof}
 $\Braket{-}$の定義と\cref{lem:func-abs}および\cref{lem:char-homo}から$\gimel 2$-値擬構造が入る事は従う.
 Boole値等号公理も普通に分解して証明すればよい. \qed
\end{proof}

$\ZFe$のモデルとしての$V^{\mathcal{R}}$では,$\hat{V}$は二値モデルではなく非自明なBoole代数である$\gimel 2$-値モデルとして振る舞っている.
一方,$V^{\mathcal{R}}$を$\ZF$のモデル$V^{\mathcal{R}}_{\ZF}$として見ると,$\gimel 2$は$\set{0, 1}$ に潰れてしまう.
もし$\truth{\varphi(-)}$が外延的同値関係$\simeq$と両立することが言えるのであれば,$\hat{V}$は$V^{\mathcal{R}}_{\ZF}$においては$\ZF$の二値的モデルとして振る舞うことになり,特に$V^{\mathcal{R}}_{\ZF}$の内部モデルとして振る舞うことが期待される.
これが成り立つかどうかはまだわからない.
\nocite{Kamo:2007}
\end{document}

% Local Variables:
% mode: yatex
% TeX-master: "realisability.tex"
% End:
