%#!luajitlatex -src-specials 05-model-of-threads.tex

\documentclass[realisability.tex]{subfiles}

\begin{document}
\section{具体例:スレッドモデル}
以下では,まず標準的な実現可能性モデルについて考える:
\begin{definition}
 $\mathcal{A}$が$\Pi_0$をスタック定数,$\Lambda_0 (\supseteq \set{\B, \C, \K, \W, \I, \cc})$を原子項とする\emph{標準実現可能性代数}$\defs$ $\Lambda$は$\Set{\conti_\pi | \pi \in \Pi} \cup \Lambda_0$の元を原子コンビネータとする$\CL$-項の全体,$\Pi$の元は$\xi_i \in \Lambda\;(i < n)$と$\pi_0 \in \Pi_0$により$\xi_0 \push \dots \push \xi_{n-1} \push \pi_0$の形のみであり,$\redsto$はKrivine機械の簡約規則に対応し,$\Lambda \cons \Pi = \Lambda \times \Pi$, $\PT \defeq \Set{ \theta \in \Lambda | \theta \text{ に } \conti_\pi \text{ の形の項が現れない}}$.
 また$\rhd$の表示は関数的.
\end{definition}

\emph{スレッドモデル}(\emph{model of threads})はKrivine~\cite{Krivine:2012hl}が定義した一番簡単な斉一的な実現可能性代数である:

\begin{definition}
 スレッド代数$\mathcal{T}$は次を満たす標準的実現可能性代数とする:
 \begin{itemize}
  \item 原子項は$\Set{\B, \C, \K, \I, \W, \cc, \varsigma}$,
  \item 相異なる可算個のスタック定数$\Set{\pi_n | n < \omega}$を持つ,
  \item $\Lambda \ni \eta \mapsto n_\eta \in \NN$を(証明的とは限らない)項の再帰的な列挙,$n \mapsto \xi_n$をその逆関数とし,$n \mapsto \theta_n$を証明的項$\PT$の再帰的な列挙とする,
  \item $\KVM{\varsigma \xi \eta \pi} \reds \KVM{{\xi}{{\num{n}_\eta}}{\pi}}$,
  \item $\pole$を次で定める:
        \[
        \xi \cons \pi \notin \pole \defs \exists n < \omega\: \theta_n \cons \pi_n \redsto \xi \cons \pi.
        \]
        $\theta_n \cons \pi_n$の簡約過程に現れる項全体を\emph{$n$番目のスレッド}と呼ぶ.
        つまり,$\pole$は何如なるスレッドにも属さないプロセスの集合である.
 \end{itemize}
\end{definition}
ソフトウェア工学において,スレッドとは同時並行で計算を行う際に使われる概念で,逐次的な計算を行う独立した単位のことである.
\begin{lemma}
 スレッド代数$\mathcal{T}$は斉一的である.
\end{lemma}
\begin{proof}
 任意の$\xi \in \PT$を取れば,$\xi = \theta_n$なる$n$があり,定義より$\theta \cons \pi_n \in \pole$となる. \qed
\end{proof}
スレッド代数の濃度は可算なので\cref{cor:dc-suff-cond}より従属選択公理が成り立つ:
\begin{lemma}
 $V^{(\mathcal{T})} \models \DC$.
\end{lemma}
本小節の目標は次の独立性証明である:

\begin{theorem}[Krivine]\label{thm:patho-sets-of-reals}
 \[
  V^{(\mathcal{T})}
   \models \ZF + \DC
    \begin{aligned}[t]
     &+\exists \Set{ \mathcal{X}_n \subseteq \Pow(\NN) | n < \omega } \:
     \forall n, m \in \NN \:
    \left[|\mathcal{X}_{n}| < |\mathcal{X}_{n+1}| \land |\mathcal{X}_n \times \mathcal{X}_m| = |\mathcal{X}_{nm}|\right],\\
     &+ \exists \Set{ \mathcal{Y}_n \subseteq \Pow(\NN) | n <\omega} \: \forall n \in\NN \:
       |\mathcal{Y}_{n+1}| < |\mathcal{Y}_n|.
    \end{aligned}
 \]
\end{theorem}

そこで,上を満たす$\mathcal{X}_n$や$\mathcal{Y}_n$たちの構成について以下では見ていく.

\subsection{$\gimel 2$のBoole構造と$\gimel n$}
以下では,von Neumann順序数もどきの表現を使って分析をする.
特に,$k < n$なる$s^k 0$の形の数項全体に太字の表記を用いる.

$\gimel$は$V^{(\mathcal{R})}$における$\hat{x}$に対応するものだったが,今考えている$V^{\mathcal{R}}$では使わないので,お行儀が悪いがこの記号を再利用する.
\begin{notation}
 $V$において$\mathbf{n} \defeq \set{0, 1, \dots, n}$として,$\gimel : \Set{ s^n 0 | n < \omega } \to \hat{V}$を次で定める:
 \[
  \gimel(s^n 0) \defeq \Set{ (s^k 0, \pi) | k < n, \pi \in \Pi } = \hat{\mathbf{n}}.
 \]
\end{notation}

$\gimel n$は,期待通り自然数の構造を反映した性質を持っていることがわかる:
\begin{lemma}\label{lem:gimel-n-props}
 次の論理式が$V^{\mathcal{R}}$で成立:
 \begin{enumerate}
  \item \label{item:lt-elem}$\forall m \strin \hat{\N}\: \forall n \strin \hat{\N}\: \left[ \truth{m < n} = 1 \iff m \strin \gimel n\right]$,
  \item \label{item:lt-str-subs}$\forall m \strin \hat{\N}\: \forall n \strin \hat{\N}\: \left[ \truth{m < n} = 1 \to \gimel m \sqsubseteq \gimel n\right]$,
  \item \label{item:lt-plus}$\forall m \strin \hat{\N}\: \forall n \strin \hat{\N}\: \left[ \truth{m < n} = 1 \iff \exists y \strin \hat{\N}\: n = m \mathbin{\hat{+}} y \mathbin{\hat{+}} 1\right]$.
 \end{enumerate}
 よって特に$\gimel n \simeq \Set{x \strin \hat{\N} | \truth{x < n} = 1}$.
\end{lemma}
\begin{proof}
 \ref{item:lt-elem} $n, m \in \N$を任意にとれば$\|\truth{\hat{m} < \hat{n}} = 1\| = \|\hat{m} \strin \gimel n\|$.
 \ref{item:lt-str-subs} $<^{\mathcal{R}}$の推移性(\cref{lem:nat-ord-in-NN})および\ref{item:lt-elem}より.
 \ref{item:lt-plus}も同様. \qed
\end{proof}
この$\gimel n$たちが,実は上の命題の$\mathcal{X}_n$たちと同じ役割を果す.
\begin{lemma}\label{lem:natural-bij-prod}
 $V^{\mathcal{R}}$において,写像$(x, y) \mapsto my + x$は$\gimel m \times \gimel n \to \gimel(mn)$への全単射となっている.
 特に,次の論理式が$\I$で実現されている:
 \begin{enumerate}
  \item $\forall m, n \strin \hat{\N}\: \forall x \strin \gimel m\: \forall y \strin \gimel n\: (my \mathbin{\hat{+}} x) \strin \gimel mn$,
  \item $\forall m, n \strin \hat{\N}\:
         \forall x, x' \strin \gimel m \: \forall y, y' \strin \gimel n\:
         \left[ my \mathbin{\hat{+}} x = m'y \mathbin{\hat{+}} x' \hookrightarrow x = x'\land y = y' \right]$,
  \item $\forall m,n \strin \hat{\N}\:\forall z \strin \gimel(mn) \:
         \exists x \strin \gimel m \: \exists y \strin \gimel n\:
         z = my \mathbin{\hat{+}} x$.
 \end{enumerate}
\end{lemma}

\begin{definition}[可算実現可能性代数の特性関数]
 $V$において$\Delta: \Set{ \hat{n} | n \in \N } \to 2$を次で定める:
 \[
  \Delta(\hat{n}) \defeq
  \begin{cases}
   0 & \text{if } \xi_n \Vdash \bot\\
   1 & \ow
  \end{cases}
 \]
\end{definition}
これを用いると,$V^{\mathcal{R}}$における$\gimel 2$の可算稠密集合が得られる:
\begin{lemma}
 $\theta \defeq \abs{xy}{\varsigma y x x}$とおくと$\theta \Vdash \forall x \strin \gimel 2\:[x \neq 0 \to \exists n \strin \NN\:(0 \neq \Delta(n) \leq x)]$.
\end{lemma}
\begin{proof}
 実際には次を示せばよい:
 \[
  \theta \Vdash \forall x \strin \gimel 2\:[x \neq 0 \to \forall n \strin \NN\:(0 \neq \Delta(n) \to \Delta(n) + a \neq a) \to \bot]
 \]
 そこで$a = 0, 1$の時に$\xi \Vdash a \neq 0$, $\eta \Vdash \forall n \strin \NN\:(\Delta(n) \neq 0 \to \Delta(n) + a \neq a)$, $\pi \in \Pi$を固定し,$\KVM{{\theta}{\xi}{\eta}{\pi}} \in \pole$を示す.
 ここで,
 \[
  \KVM{{\theta}{\xi}{\eta}{\pi}}
 \reds \KVM{{\varsigma}{\eta}{\xi}{\xi}{\pi}}
 \reds \KVM{\eta {{\num{n}_\xi}} \xi \pi}.
 \]
 $\eta$の取り方より,$\KVM{{{\num{n}_\xi}} \xi \pi} \in \|\forall n \strin \NN\: (0 \neq \Delta(n) \to \Delta(n) + a \neq a) \|$を示せばよい.
 特に$\forall n \strin \NN$の解釈の定義から,$\xi \push \pi \in \|\Delta(n_\xi) \neq 0 \to \Delta(n_\xi) + a \neq a\|$が言えればよい.
 そこで$\xi \Vdash \Delta(n_\xi) \neq 0$および$\|\Delta(n_\xi) \nleq a\| = \|\bot\| = \Pi$を示したい.
 まず,$\Delta(n_\xi) = 0$の時は定義より$\xi = \xi_{n_\xi} \Vdash \bot$かつ$\|\Delta(n_\xi) \neq 0\| = \|\bot\|$なので良い.
 $\Delta(n_\xi) = 1$の時はそもそも$\|\Delta(n_\xi) \neq 0\| = \|\top\|$となるので$\xi \in \Lambda = \|\top\|$より良い.

 あとは$\|\Delta(n_\xi) \nleq a\| = \|\bot\|$を示せばよいが,これは$\nleq$の定義より$V \models \Delta(n_\xi) \leq a$となる事が言えれば良いだけである.
 $a = 1$の時は自明.
 $a = 0$なら$\xi \Vdash a \neq 0$より$\xi \Vdash \bot$となるから$\Delta(n_\xi) = 0$となりこれもよい.
 よって示せた. \qed
\end{proof}

これを使うと,$\gimel 2$から$\Pow(\R)$への外延的埋め込みが得られることがわかる:

\begin{lemma}\label{lem:D-base-ext-inj}
 $V^{\mathcal{R}}$において対応$\gimel 2 \strni x \mapsto D_x \defeq \Set{n \strin \NN | 0 \neq \Delta(n) \leq x}$は$\gimel2$から$\Pow(\R)$への外延的埋め込みを与える.
\end{lemma}
\begin{proof}
 相異なる$x, y \strin \gimel 2$を取る.
 特に$x \cdot y = 0$である場合について考えよう.
 そこで$0 < \Delta(n) \leq x + y$なる$n \strin \NN$を取れば,$x, y$が両立しないことから$\Delta(n) \leq x \iff \Delta(n) \nleq y$が成り立つ.
 よって$n \in D_x \symdiff_{\text{ext}} D_y$となるので,$D_x \not\simeq D_y$.
 また,$x \cdot y > 0$の時は$x' = x \cdot (- y), y' y \cdot (- x)$とおいて$0 < \Delta(n) \leq x' + y'$となるものを取れば同様の事が成り立つ. \qed
\end{proof}
よって$\gimel 2$自体は$\ZF$のモデルとして見ると$\set{0, 1}$に縮退してしまうが,$\ZFe$における$\gimel 2$と同じ構造をした実数の集合が$\ZF$の側でも得られている.

ではどのような時に$\gimel 2$が非自明になるのか?という十分条件が次である:
\begin{lemma}\label{lem:gimel2-nontriv}
 証明的項$\omega_0$および$\omega_1$で次を満たすものがあるとする:
 \[
  \forall \pi \in \Pi \: \omega_0 \conti_\pi \Vdash \bot \text{ or } \omega_1 \conti_\pi \Vdash \bot.
 \]
 この時,$\gimel 2$は$V^{(\mathcal{R})}$で非自明となっている.
 特に,$\theta \defeq \abs{f}{\cc (\abs{k}{f(\omega_1 \conti)(\omega_0 \conti)})}$とおけば,
 \[
  \theta \Vdash \neg \forall x \: (x \neq 0 \to x \neq 1 \to x \strnin \gimel 2).
 \]
\end{lemma}
\begin{proof}
 $\xi \Vdash \forall x \: (x \neq 0 \to x \neq 1 \to x \strnin \gimel 2)$および$\pi \in \Pi$を任意にとって,$\theta \cons \xi \push \pi\in \pole$を示す.
 この時,
 \[
  {\theta \cons \xi \push \pi}
  \reds {\xi \cons {\omega_1 \conti_{\pi}} \push {\omega_0 \conti_{\pi}} \push \pi}.
 \]
 $\xi$に関する仮定から,任意の$x < 2$について$\xi \Vdash x \neq 0 \to x \neq 1 \to x \strnin \gimel 2$.
 よって特に$\xi \Vdash \bot \to \top \to \bot$かつ$\xi \Vdash \top \to \bot \to \bot$が成り立つ.
 $\|\top\| = \Lambda$に注意すれば,$\omega_0 \conti_\pi \Vdash \bot$でも$\omega_1 \conti_\pi \Vdash \bot$でも上の式は$\pole$に属することがわかる. \qed
\end{proof}

更にアトムを持たない十分条件が以下である:

\begin{lemma}\label{lem:gimel2-atomless}
 $\alpha_0, \alpha_1, \alpha_2 \in \PT$で任意の$\xi \in \Lambda, \pi \in \Pi$について$\conti_\pi \xi \alpha_0, \conti_\pi \xi \alpha_1$または$\conti_\pi \xi \alpha_2$のいずれかが$\perp$を実現するような物が存在するとする.
 このとき$\gimel2$はアトムを持たない.
 特に$\theta \defeq \abs{xy}{\cc (\abs{k}{x (k y \alpha_0)(x (k y \alpha_1)(k y \alpha_2))})}$とおけば,
 \[
  \theta \Vdash \forall x \strin \gimel 2 \left[ \forall y \strin \gimel 2 \: (x \cdot y > 0 \implies x \nleq y) \to x = 0 \right].
 \]
\end{lemma}
\begin{proof}
 結論部分の論理式は,$[\quad]$内の対偶を取れば次の形になる:
 \[
  \forall x \strin \gimel 2 \: [ x > 0 \to \exists y \strin \gimel 2 \: (0 < x \cdot y < x)].
 \]
 そこで適当な$x > 0$を取れば,$0 < x \cdot y < x$となる$y$が存在する.
 ここでもし$x \cdot (- y) = 0$とすると$x \cdot y = x$となり結論に反するから,$x \cdot y$および$x \cdot (- y)$は共に非零で互いに両立しない$x$未満の元になっている.
 よって上の結論が示されれば$\gimel 2$は$V^{\mathcal{R}}$でアトムを持たない事がわかる.

 以下主張を示していく.
 定義に従って$x, y$にそれぞれ$0, 1$を代入して考えれば,結局以下が言えれば良いことがわかる:
 \begin{enumerate}
  \item $\xi \Vdash \bot \to \bot \to \bot, \eta \Vdash \bot, \pi \in \Pi \implies \KVM{\theta \xi \eta \pi} \in \bot$,
  \item $\xi \Vdash \bot \to \top \to \bot, \xi \Vdash \top \to \bot \to \bot$, $\eta \in \Lambda, \pi \in \Pi \implies \KVM{\theta \xi \eta \pi} \in \pole$.
 \end{enumerate}
 簡約すれば,いずれの場合も$\KVM{\xi {{\conti_\pi \eta \alpha_0}}{\xi (\conti_\pi \eta \alpha_1)(\conti_\pi \eta \alpha_2)} \pi} \in \pole$を示せばよい.
 \begin{enumerate}
  \item $\pi \in \|\bot\|$より任意の$\psi$に対し$\conti_\pi \Vdash \bot \to \psi$であり,今$\eta \Vdash \perp$より,$\conti_\pi \eta$は任意の型を持ちうることに注意すれば明らか.
  \item $|\top| = \Lambda$より$k_\pi \eta \alpha_0 \Vdash \perp$の時は$\xi \Vdash \bot \to \top \to \bot$を使えば良い.
        そこで$k_\pi \eta \alpha_0 \nVdash \perp$とする.
        この時,一番外側の$\xi$の型は必然的に$\xi \Vdash \top \to \bot \to \bot$を使うしかないので,$\xi (\conti_\pi \eta \alpha_1) (\conti_\pi \eta \alpha_2) \Vdash \bot$が言えればよい.
        しかし仮定より$\conti_\pi \eta \alpha_1$または$\conti_\pi \eta \alpha_2$の少なくとも一方は$\perp$を実現するから,仮定より明らかに成り立つ. \qed
 \end{enumerate}
\end{proof}

いま我々は$\gimel 2$にはBoole代数の構造を入れて考えているが,$n > 2$の場合の$\gimel n$にはBooleではないにせよ環の構造を持つことがわかる.
特に,Boole代数には通常下限を積として自然に環構造が定まる.
そこで以下の定義をする:

\begin{definition}
 任意の$a \in \gimel 2$, $n \in \NN$に対し,$\gimel n$上のイデアル$a\gimel n$を次で定める:
 \[
  x \strin a \gimel n \defs x \mathbin{\hat{\cdot}} a = x.
 \]
 特に$n = 2$の場合,これは$a$による切断$\downarrow a = \Set{x \strin \gimel 2| x \leq a}$と一致する.

 また,等式に関する初等性より$\gimel 2$の各元は$\gimel n$の中で冪等元になっているので,各$a \strin \gimel 2$に対し$x \mapsto a x$により自然に写像$\gimel n \to a \gimel n$が引き起こされる.
\end{definition}

まず,$\gimel 2^n$は$\Pow(\R)$に外延的に埋め込むことが出来る:
\begin{lemma}
 $V^{\mathcal{R}} \models \forall n \strin \NN \: \gimel(2^n) \eembto \Pow(\R)$.
\end{lemma}
\begin{proof}
 まず\cref{lem:natural-bij-prod}を$n \strin \NN$についての帰納法で繰り返し適用すれば,$\gimel(2^n)$と$(\gimel 2)^n$の間に自然な全単射が存在することがわかる.
 すると,$(\gimel 2)^n \strni (b_1, \dots, b_n) \mapsto D_{b_1} \times \dots \times D_{b_n} \sqsubseteq \Pow(\NN^n)$.
 \cref{lem:D-base-ext-inj}の議論より$D_{b_1} \times \dots \times D_{b_n} \simeq D_{b'_1} \times \dots \times D_{b'_n}$なら$b_1 = b'_1, \dots, b_n = b'_n$が言えるので,この対応による行き先は外延的に相異なる.
 $\NN^n$と$\NN$の間の再帰的な全単射は\cref{lem:recursives-defined}より自然に$V^{\mathcal{R}}$の$\NN$にも持ち上がり,特にこれらは外延的全単射になっているので,\cref{lem:compos-ext-emb}より所望の結果が得られる. \qed
\end{proof}

\begin{corollary}\label{cor:gimels-ext-embeds}
 $V^{\mathcal{R}} \models \forall n \strin \NN\: \gimel n \eembto \Pow(\R)$.
\end{corollary}
\begin{proof}
 $n < 2^m$なる$m$を取れば,\cref{lem:gimel-n-props}の\ref{item:lt-str-subs}より$\gimel n \sqsubseteq \gimel (2^m) \eembto \Pow(\R)$.
\end{proof}
さて,\cref{thm:patho-sets-of-reals}で求める集合族のうち$\Set{\mathcal{X}_n | n < \omega}$は$\Set{\gimel n | n \strin \NN}$の$\Pow(\R)$における像である.
更に,(非/外延的)従属選択公理が成り立つなら,$\gimel 2$の真の無限減少列$\Braket{b_n | n < \omega}$が取れる.
実はこの$b_n$たちの定める切断が\cref{thm:patho-sets-of-reals}の$\Set{\mathcal{Y}_n | n < \omega}$に対応する.

これらが増加・減少列になっていることは明らかであり,それらは\cref{cor:gimels-ext-embeds}と\cref{cor:ext-emb-morphs}から$\ZF$においても$\Pow(\R)$の増加・減少列を生み出す.
したがって,あとは$\gimel n$達や$\downarrow b$達が真に単調であることがわかればよい.
それには次の補題を使う:

\begin{lemma}\label{lem:gimel-n-incr}
 証明的項$\omega \in \PT$で任意の$\xi, \xi' \in \Lambda$, $\xi \neq \xi'$および$\pi \in \Pi$に対し$\omega \conti_\pi \xi \Vdash \perp$または$\omega\conti_\pi \xi' \Vdash \perp$が成り立つものが取れたとする.
 この時$\theta \defeq \abs{x x'}{\cc (\abs{\conti}{\conti (x'(\abs{z}{xzz(\omega \conti z)}))})}$とおけば,任意の論理式$\varphi[x,y,z]$に対し,
 \[
  \theta \Vdash \forall m, n \strin \hat{\NN}\: \forall z \:
   \left[\begin{split}
          \braket{m < n} = 1 &\hookrightarrow
          \forall x, y, y' \left( \varphi[x,y,z] \to \varphi[x, y' z] \to y = y'\right)\\
         &\to \exists y \strin \gimel n \: \forall x \strin \gimel m \: \neg \varphi[x,y,z]
  \end{split}\right].
 \]
 特に$\varphi[x, y, z] \equiv (x, y) \in z$とおけば,$V^{\mathcal{R}}$において$m < n$なら$\gimel m$から$\gimel n$への全射は存在しない.
\end{lemma}
\begin{proof}
 背理法により示す.
 特に,$\theta$が上を実現しないということは,$m < n \in \N$と$c \in V$および
 \begin{itemize}
  \item $\xi \Vdash \forall x, y, y' \:[\varphi(x, y, c) \to \varphi(x, y', c) \to y \neq y \to \perp]$,
  \item $\xi' \Vdash \forall y \strin \gimel n \: \neg \forall x \strin \gimel m \: \neg \varphi(x, y, c)$,
  \item $\pi \in \Pi$
 \end{itemize}
 で$\KVM{\theta \xi {\xi'} \pi} \notin \pole$を満たすものが取れるという事である.
 $\pole$は上に閉じているから,$\Lambda \setminus \pole$ は下に閉じているので,結局$\KVM{{\xi'} {{(\abs{z}{\xi z z (\omega \conti_\pi z)})}} \pi } \notin \pole$となる.
 そこで$\eta \defeq \abs{z}{\xi z z (\omega \conti_\pi z)}$とおけば,$\xi'$の仮定より任意の$i < n$に対して$\eta \nVdash \forall x \strnin \gimel m\:\neg \varphi(x, i, c)$(さもなければ$\xi' \cons \eta \push \pi \in \pole$となり仮定に反する).
 そこで$\Vdash \forall$の定義により,各$i < n$に対して$m_i < m$で$\eta \nVdash  \neg \varphi(m_i, i, c)$を満たすものを固定しておく.
 すると,$\xi_i \Vdash \varphi(m_i, i, c)$と$\pi_i \in \Pi$で$\KVM{\eta {{\xi_i}} {{\pi_i}}} \notin \pole$を満たすものが取れる.
 評価を進めれば,これは$\KVM{{\xi} {{\xi_i}} {{\xi_i}} {{\omega \conti_\pi \xi_i}} {{\pi_i}}} \notin \pole$ということである.
 $\omega \conti_\pi \xi_i \Vdash i \neq i$とすると$\xi$の定義より$\eta \cons \xi_i \push \pi_i \in \pole$となってしまうので,$\omega \conti_\pi \xi_i \nVdash i \neq i$となる.
 すると$\|i \neq i\| = \|\bot\|$より任意の$i < n$について$\omega \conti_\pi \xi_i \nVdash \pole$が成り立つ.
 もし$\xi_i \neq \xi_j$なる$i < j$があれば$\omega$に関する仮定より$\omega \conti_\pi \xi_i$か$\omega \conti_\pi \xi_j$の少なくとも一つは$\perp$を実現しなくてはならないので,これは全ての$i < j < n$に対し$\xi_i = \xi_j = \xi^*$となることを意味する.
 一方,$m_i < m < n\;(i < n)$より鳩ノ巣原理から$i < j$で$m_i = m_j = m^*$となるものが少なくとも一組存在する.
 すると$\xi^* \Vdash \varphi(m^*, i, c)$かつ$\xi^* \Vdash \varphi(m^*, j, c)$.
 また$i < j$より$|i \neq j| = |\top| = \Lambda$に気を付ければ,$\omega \conti_\pi \xi^* \Vdash i \neq j$.
 すると$\KVM{\eta {{\xi^*}} {{\pi_i}}} \reds \KVM{\xi {{\xi_i}} {{\xi_i}} {{\omega \conti_\pi \xi_i}}{{\pi_i}}} \in \pole$となるが,これは$\xi_i, \pi_i$の取り方に反する. \qed
\end{proof}

\begin{corollary}
 \cref{lem:gimel-n-incr}の条件下で,$V^{\mathcal{R}}$において次を満たす実数の部分集合族$\Set{\mathcal{X}_n | n \in \NN}$が存在する:
 \[
  \aleph_0 < |\mathcal{X}_n| < |\mathcal{X}_{n+1}|, |\mathcal{X}_n \times \mathcal{X}_m| = |\mathcal{X}_{n m}|.
 \]
\end{corollary}

\begin{lemma}
 上と同じ仮定の下で,$\theta \defeq \abs{x x'}{\cc(\abs{\conti}{x(\abs{n}{\cc(\abs{\mathsf{h}}{x'\mathsf{h}\mathsf{h}(\omega \conti)(\abs{f}{f\mathsf{h}n})})})})}$とおくと,
 \[
 \theta \Vdash \forall z \left[ \forall x \strin \gimel 2 \: \exists n \strin \NN \: ( \varphi[n, x, z]) \to \exists n\: \exists x\: \exists y\: \left(\varphi[n, x, z] \land  \varphi[n, y, z] \land x \neq y\right)  \right].
 \]
 特に$V^{\mathcal{R}}$において$\NN$から$\gimel 2$への全射は存在しない.
\end{lemma}
\begin{proof}
 上と同じように$\theta$が実現しなかったとしてひたすら矛盾に向かって邁進すればいい.
 やるだけだしちょっと疲れるんで勘弁してください. \qed
\end{proof}


\subsection{スレッドモデルにおける議論}
よって,スレッドモデル$V^{\mathcal{T}}$において目標の補題を示すには,不思議な$\omega$たちの存在を証明出来ればよい.

まず,$\mathcal{T}$の構成から次が成り立つことに注意する:
\begin{lemma}\label{lem:uniq-thread}
 $\xi \in \lambda \setminus \PT$と$\pi \in \Pi$に対し,$\xi \cons \pi \notin \pole$なら$\xi \cons \pi$は一意なスレッドに現れる.
 即ち:$\exists! n \: \theta_n \cons \pi_n \reds \xi \cons \pi$.
 特に$n$は$\xi$および$\pi$に現れる一意なスタック定数$\pi_n$の番号と一致する.

 よって,$\xi \nVdash \perp$なら$\xi$はある一意なスレッドの先頭項に現れる.
\end{lemma}
\begin{proof}
 $\xi \cons \pi \notin \pole$の定義は$\xi \cons \pi$があるスレッドの実行過程に現れることだった.
 特に,$\theta_n \cons \pi_n$にはスタック定数は$\pi_n$しか現れず,簡約の過程で余分なスタック定数が追加されないので,途中に現れるプロセスは$\pi_n$以外のスタック定数を持たない. \qed
\end{proof}
また,$\mathcal{T}$は次を満たす:
\begin{theorem}[非常に強い形のChurch--Rosser性]\label{thm:strong-cr}
 $\mathcal{R}$が標準実現可能性代数で,$\xi \cons \pi \reds \xi_0 \cons \pi_0$かつ$\xi \cons \pi \reds \xi_1 \cons \pi_1$なら,$\xi_0 \cons \pi_0 \reds \xi_1 \cons \pi_1$か$\xi_1 \cons \pi_1 \reds \xi_0 \cons \pi_0$の少なくとも一方が成立する.
\end{theorem}
\begin{proof}
 $\rhd$の表示が関数的なので,簡約列は一意に定まるからあたりまえ. \qed
\end{proof}
これを踏まえると,少なくとも$V^{\mathcal{T}}$が$\gimel 2$が非自明で$\mathcal{X}_n$達が存在するようなモデルになっていることがわかる:
\begin{lemma}
 スレッド代数$\mathcal{T}$は\cref{lem:gimel2-nontriv}および\cref{lem:gimel-n-incr}の前提を満たす$\omega, \omega_0, \omega_1 \in \PT$を持つ.
\end{lemma}
\begin{proof}
 次のように置けばよい:
 \[
  \omega \defeq (\abs{x}{xx})(\abs{x}{xx}), \quad
  \omega_0 \defeq \omega \num{0}, \quad
  \omega_1 \defeq \omega \num{1}.
 \]
 まず$\omega$が期待通りであることを見る.
 $\xi, \xi' \in \Lambda$と$\pi \in \Pi$を取り$\omega \conti_\pi \xi, \omega \conti_\pi \xi' \nVdash \bot$なら$\xi = \xi'$となることを見ればよい.
 この時定義より$n, n'$および$\varpi, \varpi'$で$\theta_n \cons \pi_n \reds \omega \conti_\pi \xi' \cons \varpi$かつ$\theta_{n'} \cons \pi_{n'} \reds \omega \conti_\pi \xi' \cons \varpi'$を満たすものが取れる.
 しかし,\cref{lem:uniq-thread}より$\pi$に現れるスタック定数$\pi_n$は一意に定まるので,$n = n'$でなくてはならず,$\theta_n \cons \pi_n \reds \omega \conti_\pi \xi \cons \varpi, \omega \conti_\pi \xi' \cons \varpi'$を得る.
 すると\cref{thm:strong-cr}より$\omega \conti_\pi \xi \cons \varpi \reds \omega \conti_\pi \xi' \cons \varpi'$として一般性を失わない.
 しかし,
 \begin{align*}
   \KVM{{{\omega \conti_\pi \xi}} \varpi}
  \reds
  \KVM{{{\omega \conti_\pi}} \xi \varpi}
  \reds
  \KVM{{{\omega}} {{\conti_\pi}} \xi \varpi}
 \end{align*} 
 より候補は$\KVM{{{\omega \conti_\pi \xi}}{\varpi}} = \KVM{{{\omega \conti_\pi \xi'}}{\varpi'}}$の場合しか有り得ず,特に$\xi = \xi'$となる.

 $\omega_0, \omega_1$が所定の性質を満たすことも同様に示せる. \qed
\end{proof}
\end{document}

\subsection{次回予告}
$\gimel 2$がアトムを持たない場合の条件も満たされていることも示し,この場合$\gimel 2$の無限下降列が実数の集合の基数の無限下降列を産むことを見る.

% Local Variables:
% mode: yatex
% TeX-master: "realisability.tex"
% End:
