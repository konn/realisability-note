%#!luajitlatex -src-specials 03-equivalent-formulation.tex

\documentclass[realisability.tex]{subfiles}
\begin{document}
\section{同値な定式化}
本節では,より強制法に近い形で$\ZF$の実現可能性モデルの定式化を与える.
\subsection{名称を用いた定式化}
強制法では,「$\mathbb{P}$-名称」と呼ばれる特殊な項全体$V^{\mathbb{P}}$を或る種の$V$の拡張と思って議論をした.
これまでの実現可能性モデルでは名称に制限せず,記号の解釈が異なるのみで$V^{(\mathcal{R})} = V$となっていた.
以下では,$\ZF$のモデルを得る上では強制法と同様に「$\mathcal{R}$-名称」の全体だけを考えていれば十分であることを示す.

\begin{definition}[$\mathcal{R}$-名称の全体$V^{\mathcal{R}}$]
 $\mathcal{R} = (\Lambda, \Pi, \Lambda \star \Pi, \pole)$を実現可能性代数とする.
 この時\emph{$\mathcal{R}$-名称}の全体$V^{\mathcal{R}}$を順序数上の帰納法により次で定める:
 \begin{gather*}
  V_0^{(\mathcal{R})} \defeq \emptyset, \quad V_{\alpha + 1}^{(\mathcal{R})} \defeq \Pow(V_{\alpha}^{(\mathcal{R})} \times \Pi), \quad V_\gamma^{(\mathcal{R})} \defeq \bigcup_{\alpha < \gamma} V_\alpha^{(\mathcal{R})}\; \text{if } \gamma: \text{limit},\\
  V^{\mathcal{R}} \defeq \bigcup_{\alpha \in \On} V^{\mathcal{R}}_\alpha.
 \end{gather*}
\end{definition}

$V^{(\mathcal{R})}$上での$\ZFe$-論理式の解釈を真似て,以下では$V^{\mathcal{R}}$上で$\ZFe$-論理式の解釈を定める:

\begin{definition}[$V^{\mathcal{R}}$における論理式の解釈]
 $\ZFe$-論理式$\varphi$に対し,$V^{(\mathcal{R})}$における虚偽値$\|\varphi\|^* \subseteq \Pi$および真理値$|\varphi|^* \subseteq \Lambda$を次で定める:
\begin{gather*}
  |\varphi|^* \defeq \Set{ \xi \in \Lambda | \forall \pi \in \|\varphi\|^* \: \xi \cons \pi \in \pole},\\
  \|\bot\|^* \defeq \Pi,\qquad \|\top\|^* \defeq \emptyset, \\
  \|a \strnin b\|^* \defeq \Set{ \pi \in \Pi | (a, \pi) \in b},\\
  \|\varphi \to \psi\|^* \defeq \Set{ \xi \push \pi | {\xi \in |\varphi|^*, \pi \in \|\psi\|^*}},\\
  \|\forall x \: \varphi[x, \vec{b}]\|^* \defeq \bigcup_{a \in V^{\mathcal{R}}} \| \varphi[a, \vec{b}]\|^*,\\
  \| a \subseteq b \|^* \defeq \bigcup_{(c, \pi) \in a} \Set{ \xi \push \pi | \xi \Vdash^* c \notin b}
  = \bigcup_{c \in \dom(a)} \|c \notin b \implies c \strnin a\|^*\\
  \| a \notin b \|^* \defeq \bigcup_{(c, \pi) \in b} \Set{ \xi \push \xi' \push \pi | \xi \Vdash^* \quoted{a \subseteq c}, \xi' \Vdash^* \quoted{c \subseteq a}}
   = \|\forall x \: (a \simeq x \implies a \strnin b)\|^*
 \end{gather*}
 但し,$\xi \Vdash^* \varphi \defs \xi \in |\varphi|^*$とする.

 $\ZF_{\varepsilon}$の論理式$\varphi$に対して,関係$V^{\mathcal{R}} \models \varphi$を次で定める:
 \[
  V^{\mathcal{R}} \models \varphi \defs \theta \xi \in \PT\: \theta \Vdash^* \varphi \mathrel{}(\iff \PT \cap |\varphi|^* \neq \emptyset).
 \]
\end{definition}
すると,$V^{(\mathcal{R})}$の場合と同様にして,$V^{\mathcal{R}}$は古典論理のモデルであり,$\ZFe$の公理系も満たすことがわかる:
\begin{theorem}\label{thm:name-model-ZFe}
 $V^{\mathcal{R}} \models \ZFe$.
\end{theorem}

こうして得られた$V^{\mathcal{R}}$は,当然ながら$V^{(\mathcal{R})}$よりも少ないアトムを持つ.
しかし,$\ZF$部分については,全く同じ理論を満たすことが示せる.
それを正確に述べるため,$V^{(\mathcal{R})}$と$V^{\mathcal{R}}$の元の間の翻訳を次で与える:
\begin{definition}
 $\Phi: V^{(\mathcal{R})} \to V^{\mathcal{R}}, \Psi: V^{\mathcal{R}} \to V^{(\mathcal{R})}$をそれぞれ超限帰納法により次のように定める:
 \begin{align*}
  \Phi(x) &\defeq \Set{ (\Phi(y), \pi) | (y, \pi) \in x \cap (V \times \Pi)},\\
  \Psi(\dot{x}) &\defeq \Set{ (\Psi(\dot{y}), \pi) | (\dot{y}, \pi) \in \dot{x}}.
 \end{align*}
\end{definition}
$\Phi$と$\Psi$は本質的には殆んど同じだが,定義域・値域がテレコになっているので,議論の見通しを良くするために違う名前をつけた.
次はランクに関する帰納法により直ちに従う:
\begin{lemma}\label{lem:phi-of-psi}
 任意の$\dot{x} \in V^{\mathcal{R}}$に対し$\Phi(\Psi(\dot{x})) = \dot{x}$.
\end{lemma}
更に,原子論理式についてはこの翻訳によって真偽値が正確に保たれることもわかる:
\begin{lemma}\label{lem:atomic-coinc}
 $\dot{x}, \dot{y} \in V^{\mathcal{R}}$および$x, y \in V^{(\mathcal{R})}$に対し次が成立:
 \begin{align*}
  \|\dot{x} \strnin \dot{y}\|^*   &= \|\Psi(\dot{x}) \strnin \Psi(\dot{y})\|, &
  \|\dot{x} \notin \dot{y}\|^*    &= \|\Psi(\dot{x}) \notin \Psi(\dot{y})\|, &
  \|\dot{x} \subseteq \dot{y}\|^* &= \|\Phi(\dot{x}) \subseteq \Phi(\dot{y})\|, \\
  \|x \strnin y\|   &= \|\Phi(x) \strnin \Phi(y)\|^*, &
  \|x \notin y\|    &= \|\Phi(x) \notin \Phi(y)\|^*, &
  \|x \subseteq y\| &= \|\Phi(x) \subseteq \Phi(y)\|^*.
 \end{align*}
\end{lemma}
\begin{proof}
 $(\max(\rk(x), \rk(y)), \min(\rk(x), \rk(y)))$などなどの辞書式順序に関する帰納法. \qed
\end{proof}
次が直ちに従う:
\begin{corollary}
 $\I \Vdash \forall x \: \left(\Psi(\Phi(x)) \subseteq x\right),
  \forall x \: \left(x \subseteq \Psi(\Phi(x)) \right)$.
\end{corollary}

これらを使うと、次のような形で$V^{\mathcal{R}}$と$V^{(\mathcal{R})}$の「同値性」が言える:
\begin{theorem}
 任意の$\ZF$-論理式$\varphi$に対し$\theta_0, \theta_1 \in \PT$が存在し、任意の$x_1, \dots, x_n \in V$、$\dot{x}_1 \dots \dot{x}_n \in V^{\mathcal{R}}$および$\xi \in \Lambda$に対し次が成り立つ:
 \begin{align*}
  \xi \Vdash^* \varphi[\dot{x}_1, \dots, \dot{x}_n] &\implies \theta_0 \xi \Vdash \varphi[\Psi(\dot{x}_1), \dots, \Psi(\dot{x}_n)],\\
  \xi \Vdash \varphi[x_1, \dots, x_n] &\implies \theta_1 \xi \Vdash^* \varphi[\Phi(x_1), \dots, \Phi(x_n)].
 \end{align*}
 特に任意の$\ZF$-閉論理式$\varphi$に対し$V^{(\mathcal{R})} \models \varphi \iff V^{\mathcal{R}} \models \varphi$.
\end{theorem}
\begin{proof}
 $\varphi$が原子論理式のときは補題\ref{lem:atomic-coinc}より$\theta_0 = \theta_1 = \I$とすれば良い。

 $\varphi \to \psi$の形の時を考える.
\end{proof}

以上を踏まえ,以後は$V^{(\mathcal{R})}$の代わりに$V^{\mathcal{R}}$を用い,$\|\quad\|^*$と書く代わりに$\|\quad\|$と書いて議論をしていく事にする.
\end{document}

% Local Variables:
% mode: yatex
% TeX-master: "realisability.tex"
% End:
