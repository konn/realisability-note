%#!luajitlatex -src-specials 03-equivalent-formulation.tex

\documentclass[realisability.tex]{subfiles}
\begin{document}
\section{同値な定式化}
本節では,より強制法に近い形で$\ZF$の実現可能性モデルの定式化を与える.
\subsection{名称を用いた定式化}
強制法では,「$\mathbb{P}$-名称」と呼ばれる特殊な項全体$V^{\mathbb{P}}$を或る種の$V$の拡張と思って議論をした.
これまでの実現可能性モデルでは名称に制限せず,記号の解釈が異なるのみで$V^{(\mathcal{R})} = V$となっていた.
以下では,$\ZF$のモデルを得る上では強制法と同様に「$\mathcal{R}$-名称」の全体だけを考えていれば十分であることを示す.

\begin{definition}[$\mathcal{R}$-名称の全体$V^{\mathcal{R}}$]
 $\mathcal{R} = (\Lambda, \Pi, \Lambda \star \Pi, \pole)$を実現可能性代数とする.
 この時\emph{$\mathcal{R}$-名称}の全体$V^{\mathcal{R}}$を順序数上の帰納法により次で定める:
 \begin{gather*}
  V_0^{(\mathcal{R})} \defeq \emptyset, \quad V_{\alpha + 1}^{(\mathcal{R})} \defeq \Pow(V_{\alpha}^{(\mathcal{R})} \times \Pi), \quad V_\gamma^{(\mathcal{R})} \defeq \bigcup_{\alpha < \gamma} V_\alpha^{(\mathcal{R})}\; \text{if } \gamma: \text{limit},\\
  V^{\mathcal{R}} \defeq \bigcup_{\alpha \in \On} V^{\mathcal{R}}_\alpha.
 \end{gather*}
\end{definition}

$V^{(\mathcal{R})}$上での$\ZFe$-論理式の解釈を真似て,以下では$V^{\mathcal{R}}$上で$\ZFe$-論理式の解釈を定める:

\begin{definition}[$V^{\mathcal{R}}$における論理式の解釈]
 $\ZFe$-論理式$\varphi$に対し,$V^{(\mathcal{R})}$における虚偽値$\|\varphi\|^* \subseteq \Pi$および真理値$|\varphi|^* \subseteq \Lambda$を次で定める:
\begin{gather*}
  |\varphi|^* \defeq \Set{ \xi \in \Lambda | \forall \pi \in \|\varphi\|^* \: \xi \cons \pi \in \pole},\\
  \|\bot\|^* \defeq \Pi,\qquad \|\top\|^* \defeq \emptyset, \\
  \|a \strnin b\|^* \defeq \Set{ \pi \in \Pi | (a, \pi) \in b},\\
  \|\varphi \to \psi\|^* \defeq \Set{ \xi \push \pi | {\xi \in |\varphi|^*, \pi \in \|\psi\|^*}},\\
  \|\forall x \: \varphi[x, \vec{b}]\|^* \defeq \bigcup_{a \in V^{\mathcal{R}}} \| \varphi[a, \vec{b}]\|^*,\\
  \| a \subseteq b \|^* \defeq \bigcup_{(c, \pi) \in a} \Set{ \xi \push \pi | \xi \Vdash^* c \notin b}
  = \bigcup_{c \in \dom(a)} \|c \notin b \implies c \strnin a\|^*\\
  \| a \notin b \|^* \defeq \bigcup_{(c, \pi) \in b} \Set{ \xi \push \xi' \push \pi | \xi \Vdash^* \quoted{a \subseteq c}, \xi' \Vdash^* \quoted{c \subseteq a}}
   = \|\forall x \: (a \simeq x \implies a \strnin b)\|^*
 \end{gather*}
 但し,$\xi \Vdash^* \varphi \defs \xi \in |\varphi|^*$とする.

 $\ZF_{\varepsilon}$の論理式$\varphi$に対して,関係$V^{\mathcal{R}} \models \varphi$を次で定める:
 \[
  V^{\mathcal{R}} \models \varphi \defs \theta \xi \in \PT\: \theta \Vdash^* \varphi \mathrel{}(\iff \PT \cap |\varphi|^* \neq \emptyset).
 \]
\end{definition}
すると,$V^{(\mathcal{R})}$の場合と同様にして,$V^{\mathcal{R}}$は古典論理のモデルであり,$\ZFe$の公理系も満たすことがわかる:
\begin{theorem}\label{thm:name-model-ZFe}
 $V^{\mathcal{R}} \models \ZFe$.
\end{theorem}

こうして得られた$V^{\mathcal{R}}$は,当然ながら$V^{(\mathcal{R})}$よりも少ないアトムを持つ.
しかし,$\ZF$部分については,全く同じ理論を満たすことが示せる.
それを正確に述べるため,$V^{(\mathcal{R})}$と$V^{\mathcal{R}}$の元の間の翻訳を次で与える:
\begin{definition}
 $\Phi: V^{(\mathcal{R})} \to V^{\mathcal{R}}, \Psi: V^{\mathcal{R}} \to V^{(\mathcal{R})}$をそれぞれ超限帰納法により次のように定める:
 \begin{align*}
  \Phi(x) &\defeq \Set{ (\Phi(y), \pi) | (y, \pi) \in x \cap (V \times \Pi)},\\
  \Psi(\dot{x}) &\defeq \Set{ (\Psi(\dot{y}), \pi) | (\dot{y}, \pi) \in \dot{x}}.
 \end{align*}
\end{definition}
$\Phi$と$\Psi$は本質的には殆んど同じだが,定義域・値域がテレコになっているので,議論の見通しを良くするために違う名前をつけた.
次はランクに関する帰納法により直ちに従う:
\begin{lemma}\label{lem:phi-of-psi-inv}
 任意の$\dot{x} \in V^{\mathcal{R}}$に対し$\Phi(\Psi(\dot{x})) = \dot{x}$.
\end{lemma}
更に,原子論理式についてはこの翻訳によって真偽値が正確に保たれることもわかる:
\begin{lemma}\label{lem:atomic-coinc}
 $\dot{x}, \dot{y} \in V^{\mathcal{R}}$および$x, y \in V^{(\mathcal{R})}$に対し次が成立:
 \begin{align*}
  \|\dot{x} \strnin \dot{y}\|^*   &= \|\Psi(\dot{x}) \strnin \Psi(\dot{y})\|, &
  \|\dot{x} \notin \dot{y}\|^*    &= \|\Psi(\dot{x}) \notin \Psi(\dot{y})\|, &
  \|\dot{x} \subseteq \dot{y}\|^* &= \|\Phi(\dot{x}) \subseteq \Phi(\dot{y})\|, \\
  \|x \strnin y\|   &= \|\Phi(x) \strnin \Phi(y)\|^*, &
  \|x \notin y\|    &= \|\Phi(x) \notin \Phi(y)\|^*, &
  \|x \subseteq y\| &= \|\Phi(x) \subseteq \Phi(y)\|^*.
 \end{align*}
\end{lemma}
\begin{proof}
 $(\max(\rk(x), \rk(y)), \min(\rk(x), \rk(y)))$などなどの辞書式順序に関する帰納法. \qed
\end{proof}
次が直ちに従う:
\begin{corollary}\label{cor:psi-of-phi-ext-eq}
 $\I \Vdash \forall x \: \left(\Psi(\Phi(x)) \subseteq x\right),
  \forall x \: \left(x \subseteq \Psi(\Phi(x)) \right)$.
\end{corollary}

これらを使うと,次のような形で$V^{\mathcal{R}}$と$V^{(\mathcal{R})}$の「同値性」が言える:
\begin{theorem}
 任意の$\ZF$-論理式$\varphi$に対し$\theta_0, \theta_1 \in \PT$が存在し,任意の$x_1, \dots, x_n \in V$,$\dot{x}_1 \dots \dot{x}_n \in V^{\mathcal{R}}$および$\xi \in \Lambda$に対し次が成り立つ:
 \begin{align*}
  \xi \Vdash^* \varphi[\dot{x}_1, \dots, \dot{x}_n] &\implies \theta_0 \xi \Vdash \varphi[\Psi(\dot{x}_1), \dots, \Psi(\dot{x}_n)],\\
  \xi \Vdash \varphi[x_1, \dots, x_n] &\implies \theta_1 \xi \Vdash^* \varphi[\Phi(x_1), \dots, \Phi(x_n)].
 \end{align*}
 特に任意の$\ZF$-閉論理式$\varphi$に対し$V^{(\mathcal{R})} \models \varphi \iff V^{\mathcal{R}} \models \varphi$.
\end{theorem}
\begin{proof}
 $\varphi$が原子論理式のときは補題\ref{lem:atomic-coinc}より$\theta_0 = \theta_1 = \I$とすれば良い.

 $\varphi \to \psi$の形の時を考える.
 帰納法の仮定により$\theta^\varphi_i, \theta^\psi_i\;(i < 2)$を取れば,
 このとき$\theta_0 \defeq \abs{xy}{\theta^\psi_0(x(\theta^\varphi_1 y))}$, $\theta_1 \defeq \abs{xy}{\theta^\psi_1(x(\theta^\varphi_0 y))}$が求めるものである.

 最後に$\forall x \: \varphi[x, y]$の形の論理式の場合を考える.
 帰納法の仮定により$\theta^\varphi_0, \theta^\varphi_1$を取っておく.
 以下$a,b \in V$,$\dot{a}, \dot{b} \in V^{\mathcal{R}}$とする.
 $\xi \Vdash \forall x \: \varphi[x, b]$の時は補題\ref{lem:phi-of-psi-inv}と帰納法の仮定から$\theta^\varphi_1 \xi \Vdash^* \forall x \: \varphi[x, \Phi(x)]$となる.

 最後に$\xi \Vdash^* \forall x \: \varphi[x, \dot{b}]$の場合を考えよう.
 すると定義と帰納法の仮定から,任意の$\dot{x} \in V^{\mathcal{R}}$に対し$\theta^\varphi_0 \xi \Vdash \varphi[\Psi(\dot{x}), \Psi(\dot{b})]$が成り立つ.
 この事をつかって,任意の$x \in V$に対し$\theta_0 \xi \Vdash \varphi[x, \Psi(\dot{b})]$となるような$\theta_0$を見付けたい.
 いま$\varphi$は$\ZF$-論理式で$V^{(\mathcal{R})} \models \ZFe$なので,補題\ref{lem:ZF-conservative-lem}の\ref{item:ZF-fml-cong}より$\tau \Vdash \forall x \: \forall x'\: \forall y \: [x \subseteq x' \to x' \subseteq x \to \varphi[x, y] \to \varphi[x', y]]$となるような$\tau \in \PT$が取れる.
 また系\ref{cor:psi-of-phi-ext-eq}より$\I \Vdash \Psi(\Phi(x)) \subseteq x, x \subseteq \Psi(\Phi(x))$である.
 そこで$\theta_0 \defeq \abs{x}{\tau\I\I(\theta^\varphi_0 x)}$と定める.
 この時任意に$x \in V$を取れば,$\theta_0 \xi \Vdash \varphi[x, \Psi(\dot{b})]$となっている. \qed
\end{proof}

$V^{(\mathcal{R})}$における$\gimel E$の場合と同様,$V$の元に対応する$V^{\mathcal{R}}$の元が次で定まる:
\begin{definition}
 集合$x \in V$に対し,$\hat{x} \in V^{\mathcal{R}}$を$\in$-に関する帰納法により次で定める:
 \[
  \hat{x} \defeq \Set{(\hat{y}, \pi) | y \in x, \pi \in \Pi}.
 \]
 更に,定義可能クラス$E \subseteq V$に対し,量化子$\forall x \strin \hat{E}$の解釈を次で入れる:
 \[
  \| \forall x \strin \hat{E} \: \varphi(x)\| \defeq \bigcup_{x \in E} \|\varphi(\hat{x})\|.
 \]
 実際,$\forall x \strin \hat{E}\: \varphi(x)$と$\forall x\:[\neg \varphi(x) \to x \strnin \hat{E}]$が交換可能であることはすぐにわかる.
\end{definition}
$V^{(\mathcal{R})}$では$V$上の定義可能関数は$V^{(\mathcal{R})}$全域に拡張された.
これは,$V^{(\mathcal{R})}$の元が同時に二つの役割を果していたからである.
ひとつは,$V^{(\mathcal{R})}$における新たな集合の「名称」としての働きであり,もう一つは$V$における「古い」集合としての役割である.
前者は所属関係を考える時に現れ,後者は$V$上の関数の値や$\gimel E$を考える際に現れる.
Krivineのオリジナルの構成では,これらが互いに絡み合った形で定式化されているが,我々の$V^{\mathcal{R}}$においては最早モデルは$V$自身とは一致せず,これらは区別される.

\begin{definition}
 $f$を$V$上定義された関数とするとき,$V^{\mathcal{R}}$における$f$の解釈は次で定める:
 \[
  f^{V^{\mathcal{R}}}(\dot{x}) \defeq
 \begin{cases}
  \widehat{f(x)} & (\text{if } \exists x \in V \: \dot{x} = \hat{x})\\
  \emptyset & \ow.
 \end{cases}
 \]
 特に,$\braket{\varphi[\vec{x}]}$を$V$における$\ZF$-論理式$\varphi[\vec{x}]$の特性関数として,その$V^{\mathcal{R}}$上での拡張を考える.
\end{definition}
定義より次は明らか:
\begin{lemma}
 $V \models \forall x_1 \in A_1 \dots \forall x_n \in A_n \: f(\vec{x}) \in B \iff V^{\mathcal{R}} \models \forall x_1 \strin \hat{A}_1 \dots \forall x_n \strin \hat{A}_n \: f(\vec{x}) \strin \hat{B}$.

 特に$V \models f: A \to B$なら$V^{\mathcal{R}} \models f: \hat{A} \to \hat{B}$と思える.
\end{lemma}
また,$V^{\mathcal{R}}$の場合と同様に強い等号を次で導入する:
\begin{definition}
 \[
 \|\dot{t} = \dot{u} \hookrightarrow \varphi \| =
 \begin{cases}
  \|\varphi\| & (\dot{t} = \dot{u})\\
  \emptyset & (\ow).
 \end{cases}
 \]
\end{definition}
全く同じようにして,等号的なHorn節の同値性が示せる:
\begin{lemma}
 $t_i(\vec{x}), u_i(\vec{x}), t(\vec{x}), u(\vec{x})$を関数とすると:
 \begin{align*}
  &V \models \forall \vec{x}\:\left[t_1(\vec{x}) = u_1(\vec{x}) \to \dots \to t_n(\vec{x}) = u_n(\vec{x}) \to t(\vec{x}) = u(\vec{x})\right]\\
 \iff
 &V^{\mathcal{R}} \models \forall \vec{x}\:\left[t_1(\vec{x}) = u_1(\vec{x}) \hookrightarrow \dots \hookrightarrow t_n(\vec{x}) = u_n(\vec{x}) \hookrightarrow t(\vec{x}) = u(\vec{x})\right].
 \end{align*}
\end{lemma}
\begin{corollary}
 $V^{\mathcal{R}} \models \hat{2}: \text{Boole代数}$.
\end{corollary}
そこで以下では全て$V^{\mathcal{R}}$を使って議論を進めることにする.

\subsection{$V$を初等拡大する$\hat{2}$-値構造}
$\hat{2}$は関数$\braket{ \varphi[-] }$によって$V$の情報をコードしている.
実際,$V^{(\mathcal{R})}$の場合は$V^{(\mathcal{R})}$上に$\gimel 2$-値構造が入り,これがあたかも$V$の初等拡大のように振る舞う.
この状況は$V^{(\mathcal{R})}$という同じドメイン上に異なる構造が載っており,直観が掴みづらい.
我々の$V^{\mathcal{R}}$では,上で定義した$V^{\mathcal{R}}$の部分クラス$\hat{V} \subsetneq V^{\mathcal{R}}$が$V$の初等拡大のように振る舞うので,随分と見通しがよくなる.

まず,$\Braket{ - }$の値が自然な形で計算できる事を見る:
\begin{lemma}\label{lem:char-homo}
 $\Braket{ - }: \hat{V}^n \to \hat{2}$は論理式の関数として見た時,Boole代数の準同型のように振る舞う.即ち:
 \begin{gather*}
  V^{\mathcal{R}} \models \braket{\bot} = 0, \quad \forall \vec{x} \strin \hat{V}\: \braket{\varphi[\vec{x}] \to \psi[\vec{x}]} = - \braket{\varphi[\vec{x}]} + \braket{\psi[\vec{x}]},\\
  V^{\mathcal{R}} \models \forall \vec{y} \strin \hat{V} \: \braket{\forall x \: \varphi[x, \vec{y}]} = \prod_{x \in \hat{V}} \braket{\varphi[x, \vec{y}]}.
 \end{gather*}
\end{lemma}
\begin{proof}
 論理式の構成に関する帰納法で示す.
 $\braket{\bot} = 0$は明らか.

 また,基礎モデルにおいては$\Braket{ \varphi[x] \to \psi[x] } = -\Braket{\varphi[x]} + \braket{\psi[x]}$はどんな$x$についても成り立つから,等式に関する絶対性より
 \[
  V^{(\mathcal{R})} \models \forall x \strin \hat{V} \: \Braket{ \varphi[x] \to \psi[x] } = -\Braket{\varphi[x]} + \braket{\psi[x]}
 \]
 は常に成立する.

 最後に全称量化の場合を考える.
 次の二つがそれぞれ実現されていることを見ればよい:
 \begin{enumerate}
  \item $\forall x, z \strin \hat{V} \: \Braket{\forall y \: \varphi(x, y) } \leq \Braket{\varphi(x, z)}$,
  \item $\forall x \strin \hat{V} \: \forall \alpha \strin \gimel 2\:
          \left[\forall z \strin \hat{V} \: \alpha \leq \Braket{\varphi(x, z)}
            \implies \alpha \leq \Braket{\forall y \: \varphi(x,y)}
          \right]$.
 \end{enumerate}
 一つ目の条件,下界である事については$\comb{I}$が実現することがすぐにわかる.
 二つ目の条件も,実は$\comb{I}$が実現する.
 これを見るため,$x \in V$と$\alpha < 2$を任意に固定し,代入結果がすべて$\comb{I}$で実現されている事を見よう.

 $\alpha = 0$の時は,
 \[
  \|\forall z \strin \hat{V} \: 0 \leq \Braket{\varphi(\hat{x}, z)}
 \to 0 \leq \Braket{\forall y \: \varphi(\hat{x},y)}\| = \|(\bot \to \bot) \to (\bot \to \bot)\|
 \]
 なので,$\comb{I} \Vdash (\bot \to \bot) \to (\bot \to \bot)$よりOK.

 最後に$\alpha = 1$の時を考える.
 \[
  \|\forall z \strin \hat{V} \: 1 \leq \Braket{\varphi(\hat{x}, z)}
 \to 1 \leq \Braket{\forall y \: \varphi(x,y)}\| =
 \|\forall z \strin \hat{V} \:[\Braket{\varphi(\hat{x}, z)} = 1]
 \to \Braket{\forall y \: \varphi(\hat{x},y)} = 1\|
 \]
 よって,$\xi \Vdash \forall z \strin \hat{V} \: \Braket{\varphi(\hat{x},z)} = 1$と$\pi \in \|\Braket{\forall y \: \varphi(\hat{x},y)} = 1\|$を任意に取って$\comb{I} \cons \xi \push \pi \in \pole$を示す.
 ここで$V \models \forall y \: \varphi(x,y)$か否かで場合分けする.

 $V \models \forall x \: \varphi(x, y)$の時は,$\Braket{x, z} = 1$が任意の$z \in V$について成立しているから,
 \[
 \left|\forall z \strin \hat{V} \: \Braket{\varphi(x,z)} = 1\right| = \bigcap_{z \in V} \left|\Braket{\varphi(\hat{x}, \hat{z})} = 1\right| = | \bot \to \bot |.
 \]
 一方で仮定より$\pi \in\|\braket{\forall y \: \varphi(x, y)} = 1\| = \| \bot \to \bot \|$なので$\comb{I} \cons \xi \push \pi \reds \xi \cons \pi \in \pole$となる.

 $V \not\models \forall x \: \varphi(x)$の時を考える.
 この時$V \models \neg \varphi(y_0)$なる$y_0$を取っておけば,$\braket{\varphi(x, y_0)} = 0$であり,
 \[
 \left|\forall z \strin \hat{V} \: \Braket{\varphi(\hat{x},\hat{z})} = 1\right| = \bigcap_{z \in V} \left|\Braket{\varphi(\hat{x}, z)} = 1\right| \subseteq |\Braket{\varphi(\hat{x}, \hat{y}_0)} = 1| = | \top \to \bot |.
 \]
 また仮定より$\|\Braket{\forall y\: \varphi(x, y)} = 1\| = \|\top \to \bot\|$.
 よってこの場合も$\comb{I}$が実現してくれる事がわかる. \qed
\end{proof}
また,定義に戻って展開すれば直ちにつぎがわかる:
\begin{lemma}
 $V^{\mathcal{R}} \models \forall x, y \strin \hat{V}\: [ \braket{x \in y} = 1 \iff x \strin y],\quad \forall x, y \strin \hat{V}\:[\braket{x = y} = 1 \iff x = y]$.
\end{lemma}
個別の$\hat{\;}$-名称については,強い関係と外延的関係は一致する:
\begin{lemma}
 $\mathcal{R}$が斉一的で$x, y \in V$とする.
 \[
  V^{\mathcal{R}} \models \quoted{\hat{x} \in \hat{y}} \iff V \models x \in y, \qquad
  V^{\mathcal{R}} \models \quoted{\hat{x} \subseteq \hat{y}} \iff V \models x \subseteq y.
 \]
 よって特に$V^{\mathcal{R}} \models \hat{x} \simeq \hat{y} \iff V^{\mathcal{R}} \models \hat{x} = \hat{y}$.
\end{lemma}
\begin{remark}
 上の結果は$V^{\mathcal{R}} \models \quoted{\hat{V}: \text{外延的クラス}}$を意味しない!
 実際,$\hat{2}$が非自明な場合$\hat{V}$には$\check{x}$の形で書けないアトムが複数存在する.
\end{remark}
以上を踏まえると,$V$から$\hat{V}$への「初等埋め込み」は次の形で定式化出来る:
\begin{corollary}
 $V^{\mathcal{R}} \models \quoted{(\hat{V}, {\strin}, {=}): \ZF\text{ の }\hat{2}\text{-値モデル}}$.

 また$\varphi[x_1, \dots x_n]$を$\ZF$-論理式,$a_i \in V$とすると
 \[
  V \models \varphi[a_1, \dots, a_n] \iff V^{\mathcal{R}} \models \quoted{(\hat{V}, \hat{2}, {\strin}) \models \varphi[\hat{a}_1, \dots \hat{a}_n]}.
 \]
\end{corollary}
\begin{proof}
 $\Braket{-}$の定義と補題~\ref{lem:func-abs}および~\ref{lem:char-homo}から$\gimel 2$-値擬構造が入る事は従う.
 Boole値等号公理も普通に分解して証明すればよい. \qed
\end{proof}
\end{document}

% Local Variables:
% mode: yatex
% TeX-master: "realisability.tex"
% End:
