%#!xelatex -src-specials realisability.tex
\usepackage[hiragino-pron]{luatexja-preset}
\usepackage{mymacros}
\usepackage{luatexja-otf}
\usepackage{dsfont}
\usepackage{stmaryrd}
\usepackage{xparse}
\usepackage{mathtools}
\usetikzlibrary{math,intersections,positioning,calc,arrows.meta}
\DeclareMathAlphabet{\mathrsfs}{U}{rsfso}{m}{n}
\usepackage{fixme}
\usepackage[super]{nth}
\usepackage[bookmarksnumbered,pdfproducer={LuaLaTeX},%
            luatex,psdextra,pdfusetitle,pdfencoding=auto]{hyperref}
\usepackage{cleveref}
\usepackage[backend=biber, style=math-numeric]{biblatex}
\usepackage{mymacros}
\usepackage[framed,thmmarks,amsthm,thref,amsmath]{ntheorem}
\usepackage{framed}
\usepackage[normalem]{ulem}
\usepackage{bm}
\usepackage{cases}
\usepackage{graphicx}
\DeclareGraphicsRule{.tif}{png}{.png}{`convert #1 `dirname #1`/`basename #1 .tif`.png}
\usepackage{picins}
\usepackage[most]{tcolorbox}
\usepackage{footnote}
\renewcommand{\thefootnote}{\,*\arabic{footnote})}
\tcbuselibrary{breakable}
\newtcolorbox{mystdbox}[2][]{sharp corners,colframe=black,boxrule=0.5pt,colback=white,
 pad at break*=0pt,
 beforeafter skip=2em,left=6pt,right=6pt,top=1.5em,bottom=1.5em,breakable}
\makesavenoteenv[mystdthmbox]{mystdbox}
\newtcolorbox{myleftbar}[2][]{sharp corners,colframe=black,boxrule=0pt,colback=white,
 pad at break*=0pt,leftrule=3pt,
 beforeafter skip=2em,left=6pt,right=6pt,top=1.5em,bottom=1.5em,breakable}
\makesavenoteenv[mythmleftbar]{myleftbar}

%% Listing
\usepackage[inline]{enumitem}
\setlist[enumerate,1]{label=(\arabic*),font=\textup}
\setlist[enumerate,2]{label=(\alph*),font=\textup}

\addbibresource{myreference.bib}
\DeclareMathOperator{\rk}{rk}
\newcommand{\app}{\mathbin{\boldsymbol{\cdot}}}
\newcommand{\conti}{\mathsf{k}}
% \usepackage{txfonts}
\newcommand{\abs}[3][]{\lambda^{\mathsf{#1}}#2.\;#3}
\newcommand{\comb}[1]{\mathsf{\mathbf{#1}}}
\renewcommand{\emph}[1]{\textsf{\textgt{#1}}}
\newcommand{\eembto}{\hookrightarrow_{\mathrm{e}}}
\newcommand{\Perp}{\tikz[baseline] %
    {\path[draw,semithick] (-.375em,0) -- (.375em,0)%
                      (-.09em,0) -- +(0,.65em) (.09em,0) -- +(0,.65em)}}
\newcommand{\pole}{\mathord{\Perp}}
\newcommand{\push}{\mathbin{\boldsymbol{.}}}
\newcommand{\reds}{\mathrel{\rhd}}
\newcommand{\cons}{\mathbin{\star}}
\newcommand{\prog}[1]{\mathop{\mathtt{#1}}}
\newcommand{\redsto}{\reds}
\newcommand{\NN}{\mathord{\mathbb{N}^*}}
\DeclareMathOperator{\FV}{FV}
\newcommand{\NEAC}{\mathord{\mathreg{NEAC}}}
\newcommand{\ZFe}{\mathord{\ZF_{\varepsilon}}}
\newcommand{\strto}{\xrightarrow[\mathreg{str}]{}}
\newcommand{\extto}{\xrightarrow[\mathreg{ext}]{}}
\usepackage{knstackdiag}
\newcommand{\strin}{\mathrel{\varepsilon}}
\newcommand{\strni}{\mathrel{\reflectbox{\ensuremath{\varepsilon}}}}
\newcommand{\strnin}{\not\mathrel{\varepsilon}}
\newcommand{\strsubseteq}{\mathrel{\sqsubseteq}}
\newcommand{\PT}{\mathord{\mathreg{PT}}}
\newcommand{\MP}{\mathord{\textsc{MP}}}
\newcommand{\Abs}{\mathord{\textsc{Abs}}}
\newcommand{\A}{\comb{A}}
\renewcommand{\Gen}{\mathord{\textsc{Gen}}}
\newcommand{\Subs}{\mathord{\textsc{Subs}}}


\makeatletter
\NewDocumentCommand\stack{O{} +m}{
 \def\kn@tmp@s{#2}
 \ifx\@empty\kn@tmp@s%
 \begin{tikzpicture}[baseline=(current bounding box.south)]
  \node[minimum height=1.25em, inner sep=1ex,minimum width=3em, draw=black, dashed] (emp) at (0,0) {};
  \path[draw]
    ($(emp.south west)-(.5em, 0)$) -- ($(emp.south east)+(.5em, 0)$);
 \end{tikzpicture}
 \else
 \begin{tikzpicture}[baseline=(current bounding box.south)]
  \node [inner sep=0pt, outer sep=0pt] (stk) {\stackof[#1]{#2}};
  \path[draw]
    ($(stk.south west)-(.5em, 0)$) -- ($(stk.south east)+(.5em, 0)$);
 \end{tikzpicture} 
\fi
}
\newcommand{\VM}[2]{%
\tikz[baseline=(L.base)]{
    \node (L) {#1};
    \node (S) at ($(L.south east)+(.5em,0)$) [anchor=south west,inner sep=0]{\stack{#2}
    };
    \node at ($(L.east)+(.4em, -.25em)$) {$\cons$};
   }}
\makeatother

\newcommand{\HK}{\mathord{\mathbf{HK}}}
\newcommand{\CL}{\mathord{\mathbf{CL}}}
\newcommand{\HJ}{\mathord{\mathbf{HJ}}}

\newcommand{\B}{\comb{B}}
\renewcommand{\S}{\comb{S}}
\renewcommand{\C}{\comb{C}}
\renewcommand{\I}{\comb{I}}
\newcommand{\K}{\comb{K}}
\newcommand{\Y}{\comb{Y}}
\newcommand{\W}{\comb{W}}
\newcommand{\BCKW}{\comb{BCKW}}
\newcommand{\BCKWI}{\comb{BCKWI}}
\newcommand{\cc}{\comb{c\hspace{-1pt}c}}
\newcommand{\subst}[1]{\left[\begin{smallmatrix}#1\end{smallmatrix}\right]}
\newcommand{\num}[1]{\underline{#1}}

% 定理
\theoremstyle{definition}
\theoremseparator{.}
\theoremprework{\begin{mystdthmbox}{}}
\theorempostwork{\end{mystdthmbox}}
\newtheorem{theorem}{定理}[section]
\crefname{theorem}{定理}{定理}
\theoremprework{\begin{mystdthmbox}{}}
\theorempostwork{\end{mystdthmbox}}
\newtheorem{example}{例}[section]
\crefname{example}{例}{例}
\theoremprework{\begin{mystdthmbox}{}}
\theorempostwork{\end{mystdthmbox}}
\newtheorem{definition}{Def.}[section]
\crefname{definition}{定義}{定義}
\theoremprework{\begin{mystdthmbox}{}}
\theorempostwork{\end{mystdthmbox}}
\newtheorem{prop}{命題}[section]
\crefname{prop}{命題}{命題}
\theoremprework{\begin{mystdthmbox}{}}
\theorempostwork{\end{mystdthmbox}}
\newtheorem{corollary}{系}[section]
\crefname{corollary}{系}{系}
\theoremprework{\begin{mystdthmbox}{}}
\theorempostwork{\end{mystdthmbox}}
\newtheorem{question}{問}[section]
\crefname{question}{問}{問}
\theoremprework{\begin{mystdthmbox}{}}
\theorempostwork{\end{mystdthmbox}}
\newtheorem{problem}{問題}[section]
\crefname{problem}{問題}{問題}
\theoremprework{\begin{mystdthmbox}{}}
\theorempostwork{\end{mystdthmbox}}
\newtheorem{lemma}{補題}[section]
\crefname{lemma}{補題}{補題}
\theoremprework{\begin{mystdthmbox}{}}
\theorempostwork{\end{mystdthmbox}}
\newtheorem{remark}{注意}[section]
\crefname{remark}{注意}{注意}
\theoremprework{\begin{mystdthmbox}{}}
\theorempostwork{\end{mystdthmbox}}
\newtheorem{fact}{Fact}[section]
\crefname{fact}{事実}{事実}
\theoremprework{\begin{mystdthmbox}{}}
\theorempostwork{\end{mystdthmbox}}
\newtheorem{claim}{Claim}[section]
\crefname{claim}{主張}{主張}
\theoremprework{\begin{mystdthmbox}{}}
\theorempostwork{\end{mystdthmbox}}
\newtheorem{subclaim}{Subclaim}[claim]

\theoremstyle{nonumberplain}
\theoremprework{\begin{mystdthmbox}{}}
\theorempostwork{\end{mystdthmbox}}
\newtheorem{rem*}{注意}
\theoremprework{\begin{mystdthmbox}{}}
\theorempostwork{\end{mystdthmbox}}
\newtheorem{notation}{記号}
\theoremstyle{nonumberplain}
\theoremheaderfont{\itshape}
\theoremindent .5ex
\theorembodyfont{\upshape}
\theoremsymbol{\ensuremath{\Box}}
\theoremprework{\begin{mythmleftbar}}
\theorempostwork{\end{mythmleftbar}}
\theoremseparator{}
\newtheorem{subproof}{Proof}

%% その他記号
\qedsymbol{\ensuremath{\Box}}
\normalem


\makeatletter
\def\KVM#1{\KVM@init#1{\relax}}
\def\KVM@init#1#2{#1 \cons \KVM@rest#2}
\def\KVM@rest#1#2{%
\def\KVM@next{\KVM@rest{#1 \push #2}}
\ifx#2\relax
#1
\else
\expandafter\KVM@next
\fi
}
\makeatother

\title{実現可能性モデルノート}
\author{石井大海}

\usepackage{amssymb}	% required for `\mathbb' (yatex added)
\usepackage{amsmath}	% required for `\gather*' (yatex added)
\usepackage[normalem]{ulem}	% required for `\uwave' (yatex added)

% Local Variables:
% mode: yatex
% TeX-master: "realisability.tex"
% End:
